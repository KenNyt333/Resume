\documentclass[10pt,a4paper,ragged2e,withhyper]{altacv}
%% AltaCV uses the fontawesome5 and simpleicons packages.
%% See http://texdoc.net/pkg/fontawesome5 and http://texdoc.net/pkg/simpleicons for full list of symbols.

% Change the page layout if you need to
\geometry{left=1.25cm,right=1.25cm,top=1.5cm,bottom=1.5cm,columnsep=1.2cm}

% The paracol package lets you typeset columns of text in parallel
\usepackage{paracol}


% Change the font if you want to, depending on whether
% you're using pdflatex or xelatex/lualatex
% WHEN COMPILING WITH XELATEX PLEASE USE
% xelatex -shell-escape -output-driver="xdvipdfmx -z 0" sample.tex
\iftutex 
  % If using xelatex or lualatex:
  \setmainfont{Roboto Slab}
  \setsansfont{Lato}
  \renewcommand{\familydefault}{\sfdefault}
\else
  % If using pdflatex:
  \usepackage[rm]{roboto}
  \usepackage[defaultsans]{lato}
  % \usepackage{sourcesanspro}
  \renewcommand{\familydefault}{\sfdefault}
\fi

% Change the colours if you want to
\definecolor{SlateGrey}{HTML}{2E2E2E}
\definecolor{LightGrey}{HTML}{666666}
\definecolor{DarkPastelRed}{HTML}{450808}
\definecolor{PastelRed}{HTML}{8F0D0D}
\definecolor{GoldenEarth}{HTML}{E7D192}
\colorlet{name}{black}
\colorlet{tagline}{PastelRed}
\colorlet{heading}{DarkPastelRed}
\colorlet{headingrule}{GoldenEarth}
\colorlet{subheading}{PastelRed}
\colorlet{accent}{PastelRed}
\colorlet{emphasis}{SlateGrey}
\colorlet{body}{LightGrey}

% Change some fonts, if necessary
\renewcommand{\namefont}{\Huge\rmfamily\bfseries}
\renewcommand{\personalinfofont}{\footnotesize}
\renewcommand{\cvsectionfont}{\LARGE\rmfamily\bfseries}
\renewcommand{\cvsubsectionfont}{\large\bfseries}


% Change the bullets for itemize and rating marker
% for \cvskill if you want to
\renewcommand{\cvItemMarker}{{\small\textbullet}}
\renewcommand{\cvRatingMarker}{\faCircle}
% ...and the markers for the date/location for \cvevent
% \renewcommand{\cvDateMarker}{\faCalendar*[regular]}
% \renewcommand{\cvLocationMarker}{\faMapMarker*}


% If your CV/résumé is in a language other than English,
% then you probably want to change these so that when you
% copy-paste from the PDF or run pdftotext, the location
% and date marker icons for \cvevent will paste as correct
% translations. For example Spanish:
% \renewcommand{\locationname}{Ubicación}
% \renewcommand{\datename}{Fecha}


%% Use (and optionally edit if necessary) this .tex if you
%% want to use an author-year reference style like APA(6)
%% for your publication list
% \input{pubs-authoryear.tex}

%% Use (and optionally edit if necessary) this .tex if you
%% want an originally numerical reference style like IEEE
%% for your publication list
\input{pubs-num.tex}

%% sample.bib contains your publications
\addbibresource{sample.bib}
% \usepackage{academicons}\let\faOrcid\aiOrcid
\begin{document}
\name{DERRICK ALBINO L}
\tagline{Fresher Web Developer}
%% You can add multiple photos on the left or right
%\photoR{2.8cm}{Globe_High}
% \photoL{2.5cm}{Yacht_High,Suitcase_High}

\personalinfo{%
  % Not all of these are required!
  \email{connect2derricke@gmail.com}
  \phone{+91-70100-54808}
  \location{Chennai,Tamil Nadu,India}
  
  %\homepage{www.homepage.com}
  % \twitter{@twitterhandle}
  %\xtwitter{@x-handle}
  \linkedin{derrick-albino-473297200/}
  \github{KenNyt333}
  %\orcid{0000-0000-0000-0000}
  %% You can add your own arbitrary detail with
  %% \printinfo{symbol}{detail}[optional hyperlink prefix]
  % \printinfo{\faPaw}{Hey ho!}[https://example.com/]

  %% Or you can declare your own field with
  %% \NewInfoFiled{fieldname}{symbol}[optional hyperlink prefix] and use it:
  % \NewInfoField{gitlab}{\faGitlab}[https://gitlab.com/]
  % \gitlab{your_id}
  %%
  %% For services and platforms like Mastodon where there isn't a
  %% straightforward relation between the user ID/nickname and the hyperlink,
  %% you can use \printinfo directly e.g.
  % \printinfo{\faMastodon}{@username@instace}[https://instance.url/@username]
  %% But if you absolutely want to create new dedicated info fields for
  %% such platforms, then use \NewInfoField* with a star:
  % \NewInfoField*{mastodon}{\faMastodon}
  %% then you can use \mastodon, with TWO arguments where the 2nd argument is
  %% the full hyperlink.
  % \mastodon{@username@instance}{https://instance.url/@username}
}

\makecvheader
%% Depending on your tastes, you may want to make fonts of itemize environments slightly smaller
% \AtBeginEnvironment{itemize}{\small}

%% Set the left/right column width ratio to 6:4.
\columnratio{0.6}

% Start a 2-column paracol. Both the left and right columns will automatically
% break across pages if things get too long.
\begin{paracol}{2}
\cvsection{Objective}

%\cvevent{Job Title 1}{Company 1}{Month 20XX -- Ongoing}{Location}
{ Adaptable and highly motivated with hands-on experience in \\ front-end development, React-js and Java. Seeking an entry-level software
 developer role to contribute technical expertise, problem-solving skills, and a passion for continuous learning in a \\collaborative environment focused on innovation and growth.}

%\divider

%\cvevent{Job Title 2}{Company 2}{Month 20XX -- Ongoing}{Location}
%\begin{itemize}
%\item Job description 1
%\item Job description 2
%\end{itemize}

\cvsection{Skills}

%\cvsubsection{\faPython Programming Language}
\begin{itemize}
\item\small{\textbf{Languages:} Java, JavaScript, HTML, CSS }
\item\small{\textbf{Frameworks \& Libraries:} React.js, Node.js, Express.js}
\item\small{\textbf{Databases:} MongoDB}
\item\small{\textbf{Tools:} Vite, Roboflow }
\item\small{\textbf{Other:} Version Control(Git), AI Models(YOLOv9)}
\end{itemize}

\cvsection{Internships}
\cvevent{Combat Vehicles Research \& Development 

Establishment (CVRDE)}{Project Intern}{ February - May 2024}{}
\begin{itemize}
\item  Implemented an object detection system using YOLOv9 and \\Ultralytics to process real-time video data for multiple object identification.
\item  Created a custom dataset using Roboflow, which improved the model’s accuracy in detecting objects.
\end{itemize}



\cvsection{Projects}

\cvevent{E-Commerce Web Application Development}{ Full-Stack MERN Project}{November 2024}{}
\begin{itemize}
\item \textbf{Designed and Developed:}  Built a responsive e-commerce\\ platform with features like product listings, user authentication, shopping cart, and payment integration using MongoDB, \\Express.js, React, and Node.js.
\item \textbf{Optimized Performance:} Designed RESTful APIs and optimized database schema for seamless performance and enhanced user experience.
\end{itemize}

\medskip

\cvevent{ Self Driving Sensor Fusion}{ Nvidia CNN, LiDAR and RADAR, MDP}{April 2024}{}
\begin{itemize}
\item \textbf{Sensor Fusion:} Integrated data from LiDAR, cameras, radar, and ultrasonic sensors for enhanced environment perception, leveraging NVIDIA CNN for image processing.

\item \textbf{Simulation and Decision-Making:} Developed algorithms like\\ Kalman filter and MDP for data merging and driving decisions, tested in simulated environments using MATLAB, ROS, and \\CARLA.
\end{itemize}

\medskip

%\cvsection{A Day of My Life}

% Adapted from @Jake's answer from http://tex.stackexchange.com/a/82729/226
% \wheelchart{outer radius}{inner radius}{
% comma-separated list of value/text width/color/detail}
%

% use ONLY \newpage if you want to force a page break for
% ONLY the current column
\newpage

\cvsection{Publications}

%% Specify your last name(s) and first name(s) as given in the .bib to automatically bold your own name in the publications list.
%% One caveat: You need to write \bibnamedelima where there's a space in your name for this to work properly; or write \bibnamedelimi if you use initials in the .bib
%% You can specify multiple names, especially if you have changed your name or if you need to highlight multiple authors.
\mynames{Lim/Lian\bibnamedelima Tze,
  Wong/Lian\bibnamedelima Tze,
  Lim/Tracy,
  Lim/L.\bibnamedelimi T.}
%% MAKE SURE THERE IS NO SPACE AFTER THE FINAL NAME IN YOUR \mynames LIST

\nocite{*}

\printbibliography[heading=pubtype,title={\printinfo{\faBook}{Books}},type=book]

\divider

\printbibliography[heading=pubtype,title={\printinfo{\faFile*[regular]}{Journal Articles}},type=article]

\divider

\printbibliography[heading=pubtype,title={\printinfo{\faUsers}{Conference Proceedings}},type=inproceedings]

%% Switch to the right column. This will now automatically move to the second
%% page if the content is too long.
\switchcolumn

%\cvsection{My Life Philosophy}

%\begin{quote}
%``Something smart or heartfelt, preferably in one sentence.''
%\end{quote}

%\cvsection{Most Proud of}

%\cvachievement{\faTrophy}{Fantastic Achievement}{and some details about it}

%\divider

%\cvachievement{\faHeartbeat}{Another achievement}{more details about it of course}

%\divider

%\cvachievement{\faHeartbeat}{Another achievement}{more details about it of course}



%\cvsection{Languages}

%\cvskill{English}{5}
%\divider

%\cvskill{Spanish}{4}
%\divider

%\cvskill{German}{3.5} %% Supports X.5 values.

%% Yeah I didn't spend too much time making all the
%% spacing consistent... sorry. Use \smallskip, \medskip,
%% \bigskip, \vspace etc to make adjustments.
%\medskip
\cvsection{Certifications}
\begin{itemize}
\item  Web Development Course (SoloLearn)
        \href{https://www.sololearn.com/en/certificates/CC-M3ETMWR8}{
        \\ \faLink Link
        }
\medskip
\item  Introduction to JAVA (SoloLearn) 
         \href{https://www.sololearn.com/en/certificates/CC-GDE8N1KN}{
        \\ \faLink Link
        }
\medskip
\item Object Detection And MultiObjectTracking Using YoloV9 And Byte
        Tracker \\(CVRDE Internship Project)
        \href{https://www.linkedin.com/in/derrick-albino-473297200/details/certifications/1735481251142/single-media-viewer/?profileId=ACoAADNMotcBg8G0jMW8aDRGmBbf5VLHZXToK3w}{ 
        \\ \faLink Link
        }
\medskip        
\item    Full Stack Development Certification \\(by NoviTech RandD Private Limited)
        \href{https://www.linkedin.com/in/derrick-albino-473297200/details/certifications/1735481493536/single-media-viewer/?profileId=ACoAADNMotcBg8G0jMW8aDRGmBbf5VLHZXToK3w}{ 
        \\ \faLink Link
        }
\medskip
\item  MATLAB Onramp Certification 
  \\(Skill Development Centre, Panimalar \\Engineering College)
   \href{https://www.linkedin.com/in/derrick-albino-473297200/details/certifications/1735480481672/single-media-viewer/?profileId=ACoAADNMotcBg8G0jMW8aDRGmBbf5VLHZXToK3w}{
   \\ \faLink Link
   }
\end{itemize}

\medskip



\cvsection{Education}

\cvevent{ \textbf{B.E. EEE (Electrical \& Electronics Eng)}}{Panimalar Engineering College}{2020- 2024}{}
Overall CGPA : 8.77

\divider

\cvevent{ \textbf{HSC- Science}}{ Nazareth Matriculation Higher Secondary School}{ 2018 - 2020}{}
HSC Percentage : 72.5

\divider

\cvevent{\textbf{SSLC}}{ St Joseph Matriculation Higher Secondary School}{2005- 2018}{}
 SSLC Percentage : 87.6
% \divider

\medskip



\cvsection{Strengths}

% Don't overuse these \cvtag boxes — they're just eye-candies and not essential. If something doesn't fit on a single line, it probably works better as part of an itemized list (probably inlined itemized list), or just as a comma-separated list of strengths.

% The `ragged2e` document class option might cause automatic linebreaks between \cvtag to fail.
% Either remove the ragged2e option; or 
% add \LaTeXraggedright in the paragraph for these \cvtag
{\LaTeXraggedright
\cvtag{Hard-working}
\cvtag{Eye for detail}

\smallskip

\cvtag{Critical Thinking}
\par}

\smallskip

%% ...Or manually add linebreaks yourself
\cvtag{Good Communication}
\cvtag{Persistence}\\

\smallskip

\cvtag{Very Well Under Pressure}

\end{paracol}


\end{document}
